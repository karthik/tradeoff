\documentclass[]{article}
\usepackage[T1]{fontenc}
\usepackage{lmodern}
\usepackage{amssymb,amsmath}
\usepackage{ifxetex,ifluatex}
\usepackage{fixltx2e} % provides \textsubscript
% use microtype if available
\IfFileExists{microtype.sty}{\usepackage{microtype}}{}
\ifnum 0\ifxetex 1\fi\ifluatex 1\fi=0 % if pdftex
  \usepackage[utf8]{inputenc}
\else % if luatex or xelatex
  \usepackage{fontspec}
  \ifxetex
    \usepackage{xltxtra,xunicode}
  \fi
  \defaultfontfeatures{Mapping=tex-text,Scale=MatchLowercase}
  \newcommand{\euro}{€}
\fi
\ifxetex
  \usepackage[setpagesize=false, % page size defined by xetex
              unicode=false, % unicode breaks when used with xetex
              xetex]{hyperref}
\else
  \usepackage[unicode=true]{hyperref}
\fi
\hypersetup{breaklinks=true,
            bookmarks=true,
            pdfauthor={},
            pdftitle={},
            colorlinks=true,
            urlcolor=blue,
            linkcolor=magenta,
            pdfborder={0 0 0}}
\setlength{\parindent}{0pt}
\setlength{\parskip}{6pt plus 2pt minus 1pt}
\setlength{\emergencystretch}{3em}  % prevent overfull lines
\setcounter{secnumdepth}{0}

\author{}
\date{}

\begin{document}

\section{Realistic stage-structured models alter the optimum expectation
of classic life-history trade offs.}

Karthik Ram and Perry de Valpine.

Environmental Science, Policy \& Management. University of California,
Berkeley. Berkeley, CA 94720 karthik.ram@berkeley.edu

\begin{center}\rule{3in}{0.4pt}\end{center}

\subsection{Introduction}

Components of the life history of an organism, such as the time it takes
to reach maturity, stage-specific survival rate etc. are of considerable
interest to both theoretical and applied ecologists alike. Trade offs
between such traits allow organisms, especially ones with complex life
histories, to choose an optimal strategy that maximizes their
reproductive fitness over a range of environmental conditions (Houssard
et al., 1991). The diversity of strategies observed in nature are in
large part the result of of such tradeoffs between vital traits (Law,
1979). Understanding trade offs has been fundamental to understanding
conditions that favor the evolution of iteroparity versus semelarity
(Cole, 1954; Gadgil et al., 1970; Stearns, 1989). Ecologists have
examined trade-offs to solve applied problems such as the spread of
infectious diseases and pests under global change (Woodhams et al.,
2008).

Although there are various methods to understand tradeoffs, mathematical
models provide a particularly versatile and powerful way to examine the
importance of vital rates, and how trade offs between rates alter the
evolutionary trajectory of a species. Matrix population models,
especially, have proved extremeley useful both in a theoretical and
applied context (Caswell, 2001). These models can be applied to a
variety of organisms such as plants and insects that transition through
distinct development stages, but can also be applied to organisms that
mature through distinct size classes, which can be treated abstractly as
stages (Caswell, 2009). Matrix models have been provided numerous
insights into conservation issues (Morris et al., 2002, ) and
evolutionary ecology (Lande, 1982), the popularity driven primarily by
its simplicity and ease of interpretation.

Despite its popularity, matrix population models fail to incorporate an
adequate level biological realism that characteize most populations in
nature. For example, matrix models fail to capture variation in
development time among individuals both within and between stages (de
Valpine, 2009). Stage structured models typically assume that time spent
in a stage is geometrically distributed, an assumption that is not
realistic for most organisms. Therefore, ecologists looking to
incorporate more realism must look beyond simple matrix models (de
Valpine, 2009). Several authors have provided modifications to the
matrix model to overcome such limitations (Caswell, 2001) althogh not
all have been widely adopted (de Valpine, 2009). (Need to include a
fleshed out section describing the IPM generalization of the matrix
model).

These new methods and implementation techniques provide an easy-way to
incorporate variable development into matrix models and revisit
ecological theory. In particular, this new technique provides an
opportunity to reexamine classic life history trade offs, such as the
one between growth rate and survival, or fecundity and maturation rate,
under more complex assumptions. Here we reassess classic ecological
trade offs between maturation (both adult and juvenile) and growth rate
(Takada, 1995). In this paper, we investigate the role of incorporating
additional complexity in matrix population models, such as realistic
assumptions about time spent in a stage, and add further complexity in
the form of correlation among stages. We used the Lefkovitch matrix
model to describe stage structured population models with reproduction
and survival. To maintain a simple story, we also restrict our efforts
to a two-stage model (juveniles and adults) in a density-independent
case, although our analyses could easily be extended to larger number of
classes. We analyze the outcome with varying levels of compelxity, from
the simple matrix case all the way to incorporating variation in
development time within stages and correlation between stages. We then
compare how the optimal trade off strategy shifts as a result of these
new assumptions.

\subsection{Model Description}

\subsection{Literature Cited}

Caswell, H. 2001. \emph{Matrix Population Models: Construction, Analysis
and Interpretation}. Sinauer Associates Inc., Sunderland,MA.

Caswell, H. 2009. Stage, age and individual stochasticity in demography.
Oikos 118:1763--1782.

Cole, L. C. 1954. The population consequences of life history phenomena.
Quarterly Review of Biology 28:218--228.

Gadgil, M., \& Bossert, W. H. 1970. Life history consequences of natural
selection. The American Naturalist 102:52--64.

Heppell, S. S. 1998. Application of life-history theory and population
model analysis to turtle conservation. Copeia 1998:367--375.

Houssard, C., \& Escarre, J. 1991. The effects of seed weight on growth
and competitive ability of Rumex acetosella from two successional old
fields. Oecologia 86:236--242.

Lande, R. 1982. A quantitative genetic theory of life history evolution.
Ecology 63:607--615.

Law, R. 1979. Ecological determinants in the evolution of life
histories. In B. D. T. \&. L. R. T. R. M. Anderson (Ed.), Population
Dynamics (pp. 81--103). Blackwell Science, Oxford.

Morris, W. F., \& Doak, D. F. 2002. \emph{Quantitative conservation
biology: Theory and practice of population viability analysis}. Sinauer
Associates Inc., Sunderland,MA.

Stearns, S. C. 1989. Trade-offs in life-history evolution. Functional
Ecology 3:259--268.

Takada, T. 1995. Evolution of Semelparous and Iteroparous Perennial
Plants: Comparison between the Density-independent and Density-dependent
Dynamics. Journal of Theoretical Biology 173:51--60.

Woodhams, D. C., Alford, R. a, Briggs, C. J., Johnson, M., \&
Rollins-Smith, L. a. 2008. Life-history trade-offs influence disease in
changing climates: strategies of an amphibian pathogen. Ecology
89:1627--39.

de Valpine, P. 2009. Stochastic development in biologically structured
population models. Ecology 90:2889--2901.

\end{document}
